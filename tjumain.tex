% !Mode:: "TeX:UTF-8"
\def\usewhat{dvipdfmx}                              % 定义编译方式 dvipdfmx 或者 pdflatex ,默认为 dvipdfmx
                                                    % 方式编译,如果需要修改,只需改变花括号中的内容即可。
\documentclass[12pt,openany,oneside]{book}  %打印时选择 opanright,twoside
\usepackage{fancyhdr}
\usepackage{pdfpages}       %插入pdf
\usepackage{amsmath}
\usepackage{cases}
\usepackage{multirow}
\usepackage{bigstrut}
\newcommand{\tabincell}[2]{
\begin{tabular}{@{}#1@{}}#2\end{tabular}
}%
                                          % 如果论文超过60页 可以使用twoside 双面打印
% !Mode:: "TeX:UTF-8"
%  Authors: 张井   Jing Zhang: prayever@gmail.com     天津大学2010级管理与经济学部信息管理与信息系统专业硕士生
%           余蓝涛 Lantao Yu: lantaoyu1991@gmail.com  天津大学2008级精密仪器与光电子工程学院测控技术与仪器专业本科生

%%%%%%%%%% Package %%%%%%%%%%%%
\usepackage{graphicx}                       % 支持插图处理
%\usepackage[a4paper,text={150true mm,224true mm},top=35.5true mm,left=30true mm,head=5true mm,headsep=2.5true mm,foot=8.5true mm]{geometry}
%\usepackage[a4paper,text={153true mm,243true mm},top= 25true mm,left=31.8 true mm,head=6true mm,headsep=6true mm,foot=16.5true mm]{geometry}
\usepackage[a4paper,text={153true mm,243true mm},top= 2.75cm,left=3.57cm,right=2.77cm,bottom=2.54cm,head=6true mm,headsep=6.5true mm,foot=7.9true mm]{geometry}
%\usepackage{setpace}
                                            % 支持版面尺寸设置
\usepackage{titlesec}                       % 控制标题的宏包
\usepackage{titletoc}                       % 控制目录的宏包
\usepackage{fancyhdr}                       % fancyhdr宏包 支持页眉和页脚的相关定义
\usepackage[UTF8]{ctex}                     % 支持中文显示
\usepackage{color}                          % 支持彩色
\usepackage{amsmath}                        % AMSLaTeX宏包 用来排出更加漂亮的公式
\usepackage{amssymb}                        % 数学符号生成命令
\usepackage[below]{placeins}                %允许上一个section的浮动图形出现在下一个section的开始部分,还提供\FloatBarrier命令,使所有未处理的浮动图形立即被处理
\usepackage{flafter}                        % 使得所有浮动体不能被放置在其浮动环境之前,以免浮动体在引述它的文本之前出现.
\usepackage{multirow}                       % 使用Multirow宏包,使得表格可以合并多个row格
\usepackage{booktabs}                       % 表格,横的粗线;\specialrule{1pt}{0pt}{0pt}
\usepackage{longtable}                      % 支持跨页的表格。
\usepackage{tabularx}                       % 自动设置表格的列宽
\usepackage{subfigure}                      % 支持子图 %centerlast 设置最后一行是否居中
\usepackage[subfigure]{ccaption}            % 支持子图的中文标题
\usepackage[sort&compress,numbers]{natbib}  % 支持引用缩写的宏包
\usepackage{enumitem}                       % 使用enumitem宏包,改变列表项的格式
\usepackage{calc}                           % 长度可以用+ - * / 进行计算
\usepackage{txfonts}                        % 字体宏包
\usepackage{bm}                             % 处理数学公式中的黑斜体的宏包
\usepackage[amsmath,thmmarks,hyperref]{ntheorem}  % 定理类环境宏包,其中 amsmath 选项用来兼容 AMS LaTeX 的宏包
\usepackage{CJKnumb}                        % 提供将阿拉伯数字转换成中文数字的命令
\usepackage{indentfirst}                    % 首行缩进宏包
\usepackage{CJKutf8}                        % 用在UTF8编码环境下,它可以自动调用CJK,同时针对UTF8编码作了设置。
%\usepackage{hypbmsec}                      % 用来控制书签中标题显示内容
%支持算法环境
\usepackage[boxed,ruled,lined]{algorithm2e}
\usepackage{algorithmic}
%如果您的pdf制作中文书签有乱码使用如下命令,就可以解决了
\usepackage[dvipdfm, unicode,               % pdflatex, pdftex 这里决定运行文件的方式不同
            pdfstartview=FitH,
            %CJKbookmarks=true,
            bookmarksnumbered=true,
            bookmarksopen=true,
            colorlinks=false,
            pdfborder={0 0 1},
            citecolor=blue,
            linkcolor=red,
            anchorcolor=green,
            urlcolor=blue,
            breaklinks=true
            ]{hyperref}

                      % 定义本文所使用宏包
\graphicspath{{figures/}}                  % 定义所有的.eps文件在figures 子目录下

\begin{document}                           % 开始全文
\begin{CJK*}{UTF8}{song}                   % 开始中文字体使用
% !Mode:: "TeX:UTF-8"
%  Authors: 张井   Jing Zhang: prayever@gmail.com     天津大学2010级管理与经济学部信息管理与信息系统专业硕士生
%           余蓝涛 Lantao Yu: lantaoyu1991@gmail.com  天津大学2008级精密仪器与光电子工程学院测控技术与仪器专业本科生

%%%%%%%%%% Fonts Definition and Basics %%%%%%%%%%%%%%%%%
\newcommand{\song}{\CJKfamily{song}}    % 宋体
\newcommand{\fs}{\CJKfamily{fs}}        % 仿宋体
\newcommand{\kai}{\CJKfamily{kai}}      % 楷体
\newcommand{\hei}{\CJKfamily{hei}}      % 黑体
\newcommand{\li}{\CJKfamily{li}}        % 隶书
\newcommand{\yihao}{\fontsize{26pt}{26pt}\selectfont}       % 一号, 1.倍行距
\newcommand{\xiaoyi}{\fontsize{24pt}{24pt}\selectfont}      % 小一, 1.倍行距
\newcommand{\erhao}{\fontsize{22pt}{1.25\baselineskip}\selectfont}       % 二号, 1.25 倍行距
\newcommand{\xiaoer}{\fontsize{18pt}{18pt}\selectfont}      % 小二, 单倍行距
\newcommand{\sanhao}{\fontsize{16pt}{16pt}\selectfont}      % 三号, 1.倍行距
\newcommand{\xiaosan}{\fontsize{15pt}{15pt}\selectfont}     % 小三, 1.倍行距
\newcommand{\sihao}{\fontsize{14pt}{14pt}\selectfont}       % 四号, 1.0倍行距
\newcommand{\xiaosi}{\fontsize{12pt}{12pt}\selectfont}      % 小四, 1.倍行距
\newcommand{\wuhao}{\fontsize{10.5pt}{10.5pt}\selectfont}   % 五号, 单倍行距
\newcommand{\xiaowu}{\fontsize{9pt}{9pt}\selectfont}        % 小五, 单倍行距

%\CJKcaption{gb_452}
\CJKtilde  % 重新定义了波浪符~的意义
\newcommand\prechaptername{第}
\newcommand\postchaptername{章}

% 调整罗列环境的布局
\setitemize{leftmargin=3em,itemsep=0em,partopsep=0em,parsep=0em,topsep=-0em}
\setenumerate{leftmargin=3em,itemsep=0em,partopsep=0em,parsep=0em,topsep=0em}
%\setlength{\baselineskip}{20pt}
%\renewcommand{\baselinestretch}{1.38} % 设置行距

%避免宏包 hyperref 和 arydshln 不兼容带来的目录链接失效的问题。
\def\temp{\relax}
\let\temp\addcontentsline
\gdef\addcontentsline{\phantomsection\temp}

% 自定义项目列表标签及格式 \begin{publist} 列表项 \end{publist}
\newcounter{pubctr} %自定义新计数器
\newenvironment{publist}{%%%%%定义新环境
\begin{list}{[\arabic{pubctr}]} %%标签格式
    {
     \usecounter{pubctr}
     \setlength{\leftmargin}{2.5em}     % 左边界 \leftmargin =\itemindent + \labelwidth + \labelsep
     \setlength{\itemindent}{0em}     % 标号缩进量
     \setlength{\labelsep}{1em}       % 标号和列表项之间的距离,默认0.5em
     \setlength{\rightmargin}{0em}    % 右边界
     \setlength{\topsep}{0ex}         % 列表到上下文的垂直距离
     \setlength{\parsep}{0ex}         % 段落间距
     \setlength{\itemsep}{0ex}        % 标签间距
     \setlength{\listparindent}{0pt} % 段落缩进量
    }}
{\end{list}}%%%%%

\makeatletter
\renewcommand\normalsize{
  \@setfontsize\normalsize{12pt}{12pt} % 小四对应12pt
  \setlength\abovedisplayskip{4pt}
  \setlength\abovedisplayshortskip{4pt}
  \setlength\belowdisplayskip{\abovedisplayskip}
  \setlength\belowdisplayshortskip{\abovedisplayshortskip}
  \let\@listi\@listI}
\def\defaultfont{\renewcommand{\baselinestretch}{1.63}\normalsize\selectfont} %1.38
% 设置行距和段落间垂直距离
%\renewcommand{\CJKglue}{\hskip 0.96pt plus 0.08\baselineskip} %加大字间距,使每行33 个字
%\renewcommand{\CJKglue}{\hskip -0.1 pt plus 0.08\baselineskip} % 本科控制字间距,使每行 34 个汉字

\makeatother
%%%%%%%%%%%%% Contents %%%%%%%%%%%%%%%%%
\renewcommand{\contentsname}{目\qquad 录}
\setcounter{tocdepth}{2}
\titlecontents{chapter}[2em]{\xiaosi\song\bf}% 0.5
             {\prechaptername\CJKnumber{\thecontentslabel}\postchaptername\quad}{} %
             {\hspace{.5em}\titlerule*[5pt]{.}\xiaosi\contentspage}
\titlecontents{section}[4em]{\xiaosi\song} % 0.25
            {\thecontentslabel\quad}{} %
            {\hspace{.5em}\titlerule*[5pt]{.}\xiaosi\contentspage}
\titlecontents{subsection}[6em]{\xiaosi\song} %
            {\thecontentslabel\quad}{} %
            {\hspace{.5em}\titlerule*[5pt]{.}\xiaosi\contentspage}

%\renewcommand{\contentsname}{目\qquad 录}
%\setcounter{tocdepth}{2}
%\titlecontents{chapter}[2em]{\vspace{.25\baselineskip}\xiaosan\song}% 0.5
%             {\prechaptername\CJKnumber{\thecontentslabel}\postchaptername\qquad}{} %
%             {\hspace{.5em}\titlerule*[10pt]{$\cdot$}\sihao\contentspage}
%\titlecontents{section}[3em]{\vspace{.01\baselineskip}\sihao\song} % 0.25
%            {\thecontentslabel\quad}{} %
%            {\hspace{.5em}\titlerule*[10pt]{$\cdot$}\sihao\contentspage}
%\titlecontents{subsection}[4em]{\vspace{.01\baselineskip}\xiaosi\song} %
%            {\thecontentslabel\quad}{} %
%            {\hspace{.5em}\titlerule*[10pt]{$\cdot$}\sihao\contentspage}

%%%%%%%%%% Chapter and Section %%%%%%%%%%%%%%%%%
\setcounter{secnumdepth}{4}
\setlength{\parindent}{2em}
\renewcommand{\chaptername}{\prechaptername~\thechapter~\postchaptername}  %\CJKnumber
\titleformat{\chapter}{\centering\xiaosan\hei}{\chaptername}{0.5em}{}
\titlespacing{\chapter}{0pt}{30pt}{36pt}
\titleformat{\section}{\sihao\hei}{\thesection}{1em}{}%1em
\titlespacing{\section}{0pt}{18pt}{24pt}
\titleformat{\subsection}{\sihao\hei}{\thesubsection}{1em}{}
\titlespacing{\subsection}{0pt}{12pt}{15pt}
\titleformat{\subsubsection}{\xiaosi\hei}{\thesubsubsection}{1em}{}
\titlespacing{\subsubsection}{0pt}{9pt}{12pt}

%%%%%%%%%% Table, Figure and Equation %%%%%%%%%%%%%%%%%
\renewcommand{\tablename}{表} % 插表题头
\renewcommand{\figurename}{图} % 插图题头
\renewcommand{\thefigure}{\arabic{chapter}-\arabic{figure}} % 使图编号为 7-1 的格式 %\protect{~}
\renewcommand{\thetable}{\arabic{chapter}-\arabic{table}}%使表编号为 7-1 的格式
\renewcommand{\theequation}{\arabic{chapter}-\arabic{equation}}%使公式编号为 7-1 的格式
\renewcommand{\thesubfigure}{(\alph{subfigure})}%使子图编号为 (a)的格式
\renewcommand{\thesubtable}{(\alph{subtable})} %使子表编号为 (a)的格式
\makeatletter
\renewcommand{\p@subfigure}{\thefigure~} %使子图引用为 7-1 a) 的格式,母图编号和子图编号之间用~加一个空格
\makeatother


%% 定制浮动图形和表格标题样式
\makeatletter
\long\def\@makecaption#1#2{%
   \vskip\abovecaptionskip
   \sbox\@tempboxa{\centering\wuhao\song{#1\quad #2} }%
   \ifdim \wd\@tempboxa >\hsize
     \centering\wuhao\song{#1\quad #2} \par
   \else
     \global \@minipagefalse
     \hb@xt@\hsize{\hfil\box\@tempboxa\hfil}%
   \fi
   \vskip\belowcaptionskip}
\makeatother
\captiondelim{~~~~} %用来控制longtable表头分隔符

%%%%%%%%%% Theorem Environment %%%%%%%%%%%%%%%%%
\theoremstyle{plain}
\theorembodyfont{\song\rmfamily}
\theoremheaderfont{\hei\rmfamily}
\newtheorem{theorem}{定理~}[chapter]
\newtheorem{lemma}{引理~}[chapter]
\newtheorem{axiom}{公理~}[chapter]
\newtheorem{proposition}{命题~}[chapter]
\newtheorem{corollary}{推论~}[chapter]
\newtheorem{definition}{定义~}[chapter]
\newtheorem{conjecture}{猜想~}[chapter]
\newtheorem{example}{例~}[chapter]
\newtheorem{remark}{注~}[chapter]
%\newtheorem{algorithm}{算法~}[chapter]
\newenvironment{proof}{\noindent{\hei 证明:}}{\hfill $ \square $ \vskip 4mm}
\theoremsymbol{$\square$}

%%%%%%%%%% Page: number, header and footer  %%%%%%%%%%%%%%%%%

%\frontmatter 或 \pagenumbering{roman}
%\mainmatter 或 \pagenumbering{arabic}
\makeatletter
\renewcommand\frontmatter{\clearpage
  \@mainmatterfalse
  \pagenumbering{Roman}} % 正文前罗马字体编号
\makeatother


%%%%%%%%%% References %%%%%%%%%%%%%%%%%
\renewcommand{\bibname}{参考文献}
% 重定义参考文献样式,来自thu

\makeatletter
 %  \setlength{\parindent}{2em}  %首行缩进
\renewenvironment{thebibliography}[1]{%
   \chapter*{\bibname}%
   \wuhao
   \list{\@biblabel{\@arabic\c@enumiv}}%
        {\renewcommand{\makelabel}[1]{##1\hfill}
         \setlength{\baselineskip}{17pt}
         \settowidth\labelwidth{0.5cm}
         \setlength{\labelsep}{-2pt}  %0pt
         \setlength{\itemindent}{4em} %0pt
        \setlength{\leftmargin}{0pt}  %\labelwidth+\labelsep
         \addtolength{\itemsep}{-0.7em}
         \usecounter{enumiv}%
         \let\p@enumiv\@empty
         \renewcommand\theenumiv{\@arabic\c@enumiv}}%
    \sloppy\frenchspacing
    \clubpenalty4000%
    \@clubpenalty \clubpenalty
    \widowpenalty4000%
    \interlinepenalty4000%
    \sfcode`\.\@m}
   {\def\@noitemerr
     {\@latex@warning{Empty `thebibliography' environment}}%
    \endlist\frenchspacing}
\makeatother

\addtolength{\bibsep}{3pt} % 增加参考文献间的垂直间距
%\setlength{\bibhang}{2em} %每个条目自第二行起缩进的距离,这个原来不注释

% 参考文献引用作为上标出现
%\newcommand{\citeup}[1]{\textsuperscript{\cite{#1}}}
\makeatletter
    \def\@cite#1#2{\textsuperscript{[{#1\if@tempswa , #2\fi}]}}
\makeatother
%% 引用格式
\bibpunct{[}{]}{,}{s}{}{,}

%%%%%%%%%% Cover %%%%%%%%%%%%%%%%%
% 封面、摘要、版权、致谢格式定义
\makeatletter
\def\ctitle#1{\def\@ctitle{#1}}\def\@ctitle{}
\def\etitle#1{\def\@etitle{#1}}\def\@etitle{}
\def\caffil#1{\def\@caffil{#1}}\def\@caffil{}
\def\cmacrosubject#1{\def\@cmacrosubject{#1}}\def\@cmacrosubject{}
\def\cmacrosubjecttitle#1{\def\@cmacrosubjecttitle{#1}}\def\@cmacrosubjecttitle{}
\def\csubject#1{\def\@csubject{#1}}\def\@csubject{}
\def\csubjecttitle#1{\def\@csubjecttitle{#1}}\def\@csubjecttitle{}
\def\cgrade#1{\def\@cgrade{#1}}\def\@cgrade{}
\def\cauthor#1{\def\@cauthor{#1}}\def\@cauthor{}
\def\cauthortitle#1{\def\@cauthortitle{#1}}\def\@cauthortitle{}
\def\csupervisor#1{\def\@csupervisor{#1}}\def\@csupervisor{}
\def\csupervisortitle#1{\def\@csupervisortitle{#1}}\def\@csupervisortitle{}
\def\cdate#1{\def\@cdate{#1}}\def\@cdate{}
\def\declaretitle#1{\def\@declaretitle{#1}}\def\@declaretitle{}
\def\declarecontent#1{\def\@declarecontent{#1}}\def\@declarecontent{}
\def\authorizationtitle#1{\def\@authorizationtitle{#1}}\def\@authorizationtitle{}
\def\authorizationcontent#1{\def\@authorizationcontent{#1}}\def\@authorizationconent{}
\def\authorizationadd#1{\def\@authorizationadd{#1}}\def\@authorizationadd{}
\def\authorsigncap#1{\def\@authorsigncap{#1}}\def\@authorsigncap{}
\def\supervisorsigncap#1{\def\@supervisorsigncap{#1}}\def\@supervisorsigncap{}
\def\signdatecap#1{\def\@signdatecap{#1}}\def\@signdatecap{}
\long\def\cabstract#1{\long\def\@cabstract{#1}}\long\def\@cabstract{}
\long\def\eabstract#1{\long\def\@eabstract{#1}}\long\def\@eabstract{}
\def\ckeywords#1{\def\@ckeywords{#1}}\def\@ckeywords{}
\def\ekeywords#1{\def\@ekeywords{#1}}\def\@ekeywords{}

%在book文件类别下,\leftmark自动存录各章之章名,\rightmark记录节标题
\pagestyle{fancy}
%去掉章节标题中的数字 务必放到\pagestyle{fancy}之后才会起作用
%%不要注销这一行,否则页眉会变成:“第1章1  绪论”样式
\renewcommand{\chaptermark}[1]{\markboth{\chaptername~\ #1}{}}
  \fancyhf{}
  \fancyhead[C]{\song\wuhao \leftmark} % 页眉显示章节名称
  %\fancyhead[CO]{\song\wuhao \@cheading}
  %\fancyhead[CE]{\song\wuhao \@cheading}
  \fancyfoot[C]{\song\xiaowu ~\thepage~}
%  \renewcommand{\headrulewidth}{0pt} %去掉横线,正文的。

\fancypagestyle{plain}{% 设置开章页页眉页脚风格,每章第一页设置
    \fancyhf{}%
    \fancyhead[C]{\song\wuhao \leftmark}
    \fancyfoot[C]{\song\xiaowu ~\thepage~ } %%首页页脚格式
    \renewcommand{\headrulewidth}{0.5pt}%  0.5pt
    \renewcommand{\footrulewidth}{0pt}%
}

\newlength{\@title@width}
\def\@put@covertitle#1{\makebox[\@title@width][s]{#1}}
% 定义封面
\def\makecover{
%\cleardoublepage%
   \phantomsection
    \pdfbookmark[-1]{\@ctitle}{ctitle}

    \begin{titlepage}
      \vspace*{0.8cm}
      \begin{center}

      \vspace*{1cm}
      {\erhao\song\textbf{\@ctitle}}%hei
      \vspace*{1cm}

      \vspace*{1cm}

      \begin{center}
      \renewcommand{\baselinestretch}{1.6} % 设置行距
      \hei\erhao\bf{\@etitle}%hei  应该是roman
      \renewcommand{\baselinestretch}{1.38} % 设置行距
      \end{center}

      \vspace*{3cm}
      \setlength{\@title@width}{5em}
      {\song\sihao
      \begin{tabular}{p{\@title@width}@{:}l}
        \@put@covertitle{\@csubjecttitle} & \@csubject \\
        \@put@covertitle{\@cauthortitle} & \@cauthor \\
        \@put@covertitle{\@csupervisortitle} & \@csupervisor \\
      \end{tabular}
      }

  \vspace*{5cm}
  \song\sihao\@caffil \\
  \song\sihao\@cdate

\end{center}
%  另起一页: 独创性声明和学位论文版权使用授权书
\newpage
    \clearpage
    \thispagestyle{empty} %去掉页眉页脚
    \vspace*{1cm}
    \begin{center}\song\xiaoer{\@declaretitle}\end{center}\par
    \song\xiaosi{\@declarecontent}\par
    \vspace*{1cm}
    {\song\xiaosi
    \@authorsigncap \makebox[2.5cm][s]{}
    \@signdatecap \makebox[2cm][s]{} 年 \makebox[1cm][s]{} 月 \makebox[1cm][s]{} 日
    }

    \vspace*{3cm}
    \begin{center}\song\xiaoer{\@authorizationtitle}\end{center}\par
    {
    \song\xiaosi{\@authorizationcontent}

    \@authorizationadd\par
    }

    \vspace*{2cm}
    {\noindent\song\xiaosi
    \begin{tabularx}{\textwidth}{ll}
        \@authorsigncap \makebox[3.5cm][s]{}  & \@supervisorsigncap \makebox[3.5cm][s]{}   \\
         &  \\
        \@signdatecap \makebox[1.5cm][s]{} 年 \makebox[1cm][s]{} 月 \makebox[1cm][s]{} 日 &
         \@signdatecap \makebox[1.5cm][s]{} 年 \makebox[1cm][s]{} 月 \makebox[1cm][s]{} 日 \\
    \end{tabularx}
    }
\end{titlepage}

%%%%%%%%%%%%%%%%%%%   Abstract and Keywords  %%%%%%%%%%%%%%%%%%%%%%%
\clearpage
\thispagestyle{plain}% 设置开章页页眉页脚风格
\markboth{摘~要}{摘~要}
\addcontentsline{toc}{chapter}{摘~要}
\chapter*{\centering\erhao\song\textbf{摘\qquad 要}}
\setcounter{page}{1}
\song\defaultfont
\@cabstract
%\vspace{\baselineskip}   %原来有

%\hangafter=1\hangindent=52.3pt\noindent   %如果取消该行注释,关键词换行时将会自动缩进
\noindent
{\hei\sihao  关键词:} \@ckeywords  %\hei\sihao  关键词:

%%%%%%%%%%%%%%%%%%%   English Abstract  %%%%%%%%%%%%%%%%%%%%%%%%%%%%%%
\clearpage
\markboth{ABSTRACT}{ABSTRACT} %这里括号中填的就是页眉的值
%\renewcommand{\headrulewidth}{0pt} %去掉横线,但是会去掉正文的。
\addcontentsline{toc}{chapter}{ABSTRACT}
\chapter*{\centering\erhao\bf{ABSTRACT}}
%\vspace{\baselineskip}
%\thispagestyle{plain}
\@eabstract
%\vspace{\baselineskip}  %原来有

%\hangafter=1\hangindent=60pt\noindent  %如果取消该行注释,KEY WORDS换行时将会自动缩进
\noindent
{\sihao\textbf{KEY WORDS:}}  \@ekeywords
}
%\renewcommand{\headrulewidth}{0.5pt} %去掉横线,但是会去掉正文的。
\makeatother
                       % 完成对论文各个部分格式的设置
\frontmatter                               % 以下是论文导言部分,包括论文的封面,中英文摘要和中文目录
% !Mode:: "TeX:UTF-8"


\ctitle{结构性缺失低秩矩阵重建研究 \\ 及其图像处理应用}  %封面用论文标题,自己可手动断行
\etitle{Reconstruction of Structurally-Incomplete Matrices and Its Image Processing Applications}
\caffil{天津大学电子信息工程学院} %学院名称
\csubjecttitle{学科专业}
\csubject{信息与通信工程}   %专业
\cauthortitle{研究生}     % 学位
\cauthor{杨雪梦}   %学生姓名
\csupervisortitle{指导教师}
\csupervisor{杨敬钰~副教授} %导师姓名

\declaretitle{独创性声明}
\declarecontent{
本人声明所呈交的学位论文是本人在导师指导下进行的研究工作和取得的研究成果,除了文中特别加以标注和致谢之处外,论文中不包含其他人已经发表或撰写过的研究成果,也不包含为获得 {\underline{\kai\textbf{~ 天津大学~}}} 或其他教育机构的学位或证书而使用过的材料。与我一同工作的同志对本研究所做的任何贡献均已在论文中作了明确的说明并表示了谢意。
}
\authorizationtitle{学位论文版权使用授权书}
\authorizationcontent{
本学位论文作者完全了解{\underline{\kai\textbf{~天津大学~}}}有关保留、使用学位论文的规定。特授权{\underline{\kai\textbf{~天津大学~}}} 可以将学位论文的全部或部分内容编入有关数据库进行检索,并采用影印、缩印或扫描等复制手段保存、汇编以供查阅和借阅。同意学校向国家有关部门或机构送交论文的复印件和磁盘。
}
\authorizationadd{(保密的学位论文在解密后适用本授权说明)}
\authorsigncap{学位论文作者签名:}
\supervisorsigncap{导师签名:}
\signdatecap{签字日期:}


%\cdate{\CJKdigits{\the\year} 年\CJKnumber{\the\month} 月 \CJKnumber{\the\day} 日}
% 如需改成二零一二年四月二十五日的格式,可以直接输入,即如下所示
 \cdate{二零一六年十二月}
% \cdate{\the\year 年\the\month 月 \the\day 日} % 此日期显示格式为阿拉伯数字 如2012年4月25 日
\cabstract{
大多数传统矩阵重建算法都是利用矩阵的低秩特性或者相关改进先验信息来约束待重建矩阵。当观测矩阵中的缺失元素位置满足随机分布时,这些方法可以有效地对缺失元素进行恢复。然而,在实际的应用中,矩阵中缺失的元素往往不符合随机分布这一假设。相反,矩阵中缺失元素的位置常常表现出一定的结构性,而这些具有结构性缺失的矩阵则无法使用只基于低秩特性的传统矩阵重建方法来进行恢复。为了解决这一问题,本文同时使用低秩特性和稀疏特性两种先验信息,提出一类新的矩阵重建模型及其求解算法。论文的工作成果如下:

1.~提出基于重加权低秩先验和稀疏先验的结构性缺失矩阵重建模(MC-ReLaSP),克服传统方法只能处理随机缺失矩阵的缺点。该模型使用低秩先验约束待重建矩阵,从而探索矩阵行间以及列间的相关性;使用稀疏先验约束待重建矩阵,从而利用矩阵行内或列内的相关性;使用重加权策略分别对低秩特性和稀疏特性两个先验进行强化,从而提升矩阵重建精度。

2.~以~MC-ReLaSP~模型为基础,进一步考虑矩阵受不同噪声污染的情况,提出了有噪声的结构性缺失矩阵重建模型:高斯噪声矩阵重建模型(MR-ReLaSP1)和脉冲噪声矩阵重建模型(MR-ReLaSP2)。扩展模型能够对同时遭受元素缺失和元素受噪的矩阵进行重建。

3.~针对所提的三个模型,提出增广拉格朗日乘子法框架下的交替方向算法。在重加权的整体框架下,将等式约束通过增广拉格朗日乘子法转化为一系列无约束子问题。采用交替方向方法解耦子问题的多元变量,从而降低优化问题的复杂度。

4.~评测本文模型与传统低秩矩阵重建模型的恢复能力,并探索所提模型的多种图像处理应用。合成数据上的评测结果表明,所提模型对结构性缺失矩阵的恢复性能显著超越传统低秩矩阵重建方法。本文将所提模型成功应用于图像填充、图像去噪、图像去雨线等应用,并且获得出色图像恢复性能。
}

\ckeywords{低秩矩阵重建,稀疏表示,重加权,图像恢复}

\eabstract{
Most matrix reconstruction methods impose a low-rank prior or its variants to well pose the problem, which can reconstruct randomly-missing entries in a matrix efficiently. However, in practical applications, these missing entries are not distributed randomly, but represent a trend just like structural missing, which cannot be handled by the rank minimization prior individually. To remedy this, this paper introduces new matrix reconstruction models and algorithms using double priors on the latent matrix. The main contents of this thesis are summarized as follows:

1.~This paper proposed a structurally-incomplete matrix completion model (MC-ReLaSP) based on reweighted low-rank and sparsity priors, complementing the classic matrix reconstruction models that handle random missing only. In the proposed model, the matrix is regularized by a low-rank prior to exploit the inter-column and inter-row correlations, and its columns/rows are regularized by a sparsity prior to exploit intra-column/row correlations. Both the low-rank and sparse priors are reweighted on the fly to promote low-rankness and sparsity, respectively.

2.~This paper proposed two variant models, i.e. MR-ReLaSP1 model for Gaussian noise and MR-ReLaSP2 model for impulse noise, by considering both structural missing and noise in observed entries based on the MC-ReLaSP model, which enhances the robustness of our proposed models in practical applications.

3.~Numerical algorithms to solve our models are derived via the alternating direction method under the augmented Lagrangian multiplier (ALM-ADM) framework. The proposed algorithm has a low computational complexity benefiting from ADM.

4.~We evaluate the recoverability of our proposed models and classic matrix reconstruction models, and apply our models to various image processing applications. Results on synthetic data show that the proposed models outperform the classic ones. Our models are quite effective in image processing applications, such as image inpainting, image denoising and image rain-streak removing.
}

\ekeywords{Low rank matrix reconstruction, Sparse representation, Iterative reweighting, Image restoration }

\makecover

\clearpage  %加上去掉页码
                      % 封面

%%%%%%%%%%   目录   %%%%%%%%%%

\defaultfont
\clearpage{\pagestyle{plain}\cleardoublepage}  %empty
\thispagestyle{plain}
%\addcontentsline{toc}{chapter}{目~~~~录} %原来有
%\pagestyle{fancyplain} %实在没办法就用所有都没有页眉的前目录+有页眉的正文组合pdf
%\fancyhf{}
%\renewcommand{\headrulewidth}{0pt}
\tableofcontents                          % 中文目录
%%%%%%%%%% 正文部分内容  %%%%%%%%%%
\mainmatter\defaultfont\sloppy\raggedbottom
% !Mode:: "TeX:UTF-8"

\markboth{总结与展望}{总结与展望}
%\addcontentsline{toc}{chapter}{结\quad 论} %添加到目录中
\chapter{总结与展望}


\section{总结}

结论应是作者在学位论文研究过程中所取得的创新性成果的概要总结,不能与摘要混为一谈。
学位论文结论应包括论文的主要结果、创新点、展望三部分,在结论中应概括论文的核心观点,
明确、客观地指出本研究内容的创新性成果(含新见解、新观点、方法创新、技术创新、理论创新),
并指出今后进一步在本研究方向进行研究工作的展望与设想。
对所取得的创新性成果应注意从定性和定量两方面给出科学、准确的评价,分(1)、(2)、(3)…条列出,宜用“提出了”、“建立了”等词叙述。

\section{展望}
展望是对你下一步工作的简单阐述。



%\include{body/chapter2}
%\include{body/chapter3}
%\include{body/chapter4}
%\include{body/chapter5}
%\include{body/chapter6}
%%%%%%%%%% 正文部分内容  %%%%%%%%%%

%%%%%%%%%%  参考文献  %%%%%%%%%%
\defaultfont
\bibliographystyle{TJUThesis}
\phantomsection
\markboth{参考文献}{参考文献}
\addcontentsline{toc}{chapter}{参考文献}          % 参考文献加入到中文目录
\nocite{*}                                        % 若将此命令屏蔽掉,则未引用的文献不会出现在文后的参考文献中。
\bibliography{references/reference}
\include{appendix/publications}                   % 发表论文和参加科研情况说明
\include{appendix/acknowledgements}               % 致谢
\clearpage
\end{CJK*}                                        % 结束中文字体使用
\end{document}                                    % 结束全文
