% !Mode:: "TeX:UTF-8"

\markboth{总结与展望}{总结与展望}
%\addcontentsline{toc}{chapter}{结\quad 论} %添加到目录中
\chapter{总结与展望}


\section{总结}

结论应是作者在学位论文研究过程中所取得的创新性成果的概要总结,不能与摘要混为一谈。
学位论文结论应包括论文的主要结果、创新点、展望三部分,在结论中应概括论文的核心观点,
明确、客观地指出本研究内容的创新性成果(含新见解、新观点、方法创新、技术创新、理论创新),
并指出今后进一步在本研究方向进行研究工作的展望与设想。
对所取得的创新性成果应注意从定性和定量两方面给出科学、准确的评价,分(1)、(2)、(3)…条列出,宜用“提出了”、“建立了”等词叙述。

\section{展望}
展望是对你下一步工作的简单阐述。


